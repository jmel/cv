\documentclass[12pt]{article}
\setlength{\textheight}{8.75 in}
\setlength{\textwidth}{6. in}
\setlength{\oddsidemargin}{-0.2 cm}
\setlength{\topmargin}{-0.2 cm}
\usepackage{fancyhdr}
\usepackage{lastpage}
\pagestyle{fancy}
\fancyhead{}
\renewcommand{\headrulewidth}{0.pt}
\lfoot{Jason Melbourne}
\rfoot{Hubble Fellowship CV}
\cfoot{\thepage/2}
\begin{document}

\begin{center}
Curriculum Vitae\\
{\Large\bf{Jason Melbourne}}\\
\textnormal
Graduate Student\\
UCO/Lick Observatory,
University of California Santa Cruz,
Santa Cruz, CA 95064\\
831-459-5891\\
jmel@ucolick.org
\end{center}

\noindent
{\bf{Degrees}}

\begin{itemize}
\addtolength{\itemsep}{-0.25cm}

\item[2006]  Ph.D. Astronomy, University of California Santa Cruz (May 2006)\\
	 Thesis: \emph{The Optical and Infrared Evolution of Blue Galaxies to $z=1$.} (D. C. Koo, advisor)
	 
\item[2001] M.A. Astronomy, Wesleyan University\\
 Thesis: \emph{Metal Abundances in KISS Galaxies} (J. J. Salzer, advisor) 

\item[1995] B.A. Physics and Astronomy (double major), University of California Berkeley

\end{itemize}

\noindent
{\bf{Schools and Workshops}}
\begin{itemize}
\addtolength{\itemsep}{-0.25cm}

\item[2005] The Center for Adaptive Optics Professional Development Workshop

\item[2004] The Jerusalem Winter School: The Origin of Galaxies 

\item[2001] The Vatican Observatory Summer School: Compact Objects 

\end{itemize}

\noindent
{\bf{Research Experience}} \\
(See Research Summary and Publication List for Details)
\begin{itemize}
\addtolength{\itemsep}{-0.25cm}

\item[2001] -- 2005 Graduate Student Researcher, University of California Santa Cruz\\
Emphasis on extragalactic observational astronomy across multi-wavelength including UV (STIS), optical (ACS), near IR (Lick and Keck Adaptive Optics), mid IR (\emph{Spitzer} MIPS), and radio (Aricebo).  Completed four first author papers (see Publications List), advisors: D. C. Koo, C. Max, S. Faber.

\item[1999] -- 2001 Graduate Student Researcher, Wesleyan University \\
Emphasis on spectroscopy of low-z emission line galaxies.  Completed two first author papers (see Publications List), advisors: J. J. Salzer, A. Sarajedini.

\item[1995] -- 1996 Assistant Researcher, Lawrence Berkeley National Laboratory\\
The Supernova Cosmology Project, emphasis on optical photometry, advisor: Saul Perlmutter.

\item[1992] -- 1995 Undergraduate Researcher, University of California Berkeley\\
Narrow-band photometry of comets,  advisor: H. Spinrad.   

\end{itemize}

\noindent
{\bf{Successful Proposals}} \\
(In which J. Melbourne was primary contributer)
\begin{itemize}
\addtolength{\itemsep}{-0.25cm}

\item [2005] Co-Investigator. Keck Laser Guide Star Adaptive Optics Imaging of Large Disk Galaxies in GOODS-S 

\item [2005] Co-Investigator. Keck Laser Guide Star Adaptive Optics Imaging of Chandra X-ray Sources in GOODS-S

\item [2004] Co-Investigator. Keck Natural Guide Star Adaptive Optics Imaging of Galaxies in GEMS and COSMOS

\item [2003] Principle-Investigator. Lick Laser Guide Star Adaptive Optics Imaging of Local Blue Compact Galaxies  
\end{itemize}

\noindent
{\bf{Recent Talks}}
\begin{itemize}
\addtolength{\itemsep}{-0.25cm}
%\labelitemiii

\item[2005] \emph{A Practical Guide to Adaptive Optics Observing on Keck.} 
The Center for Adaptive Optics Extragalacitic AO Workshop, UC Sanata Cruz, August 2005

\item[2005] \emph{Keck Laser Illuminates AGN in the Distant Universe.} American Astronomical Society Meeting 205, January 2005

\item[2004]  \emph{Laser Illuminates Compact Galaxies.} Starbursts: From 30 Doradus to Lyman Break Galaxies, Cambridge, United Kingdom, September 2004
\end{itemize}

\noindent
{\bf{Teaching Experience}}
\begin{itemize}
\addtolength{\itemsep}{-0.25cm}
\item[2005] Lead Instructor \\ Center for Adaptive Optics Mainland Short Course for Undergraduates
\item[2003] Project Advisor \\COSMOS, an Astronomy Summer Program for High School Students
\item[1999] -- 2002 Teaching Assistant \\  Undergraduate Astronomy Classes
\item[1998] -- 1999 Astronomy Teacher \\ Project Astro, Astronomy for Elementary School Classes
\end{itemize}

\noindent
{\bf{Students Advised}}
\begin{itemize}
\addtolength{\itemsep}{-0.25cm}
\item[2005] Emily De La Garza, Center for Adaptive Optics Undergraduate summer intern\\ \emph{Stellar Population Synthesis Models of Synthetic Galaxies}.   Emily won best poster for her research at the Society for Advancement of Chicanos/Latinos and Native Americans (SACNAS) annual science meeting, September 2005.
\item[2004]  Conswella White, Center for Adaptive Optics Undergraduate summer intern\\ \emph{Decomposing Adaptive Optics Images of Disk Galaxies.}
\end{itemize}

\noindent
{\bf{Professional Service}}
\begin{itemize}
\addtolength{\itemsep}{-0.25cm}
\item[2005] Organized and Chaired the The Center for Adaptive Optics Extragalactic AO Workshop, University of California Santa Cruz, August 2005
\item[2004] The Graduate Student Representative to the UCO/Lick Observatory Job Search Committee
\item[2003] -- 2004 UC Santa Cruz Astronomy Department, Graduate Student Spokesperson
\end{itemize}


%{\bf{\Large Publications:}}\\
%\noindent
%Melbourne, J., Phillips, A., Salzer, J.~J. et al. ``Metal Abundances of KISS 
%Galaxies. II. Nebular Abundances of Twelve Low-Luminosity Emission-Line
%Galaxies.'' \ 2004, AJ (accepted)\\ 

%\noindent
%Melbourne, J.~\& Guhathakurta, P. ``Measuring the Slope of the Galactic
%Dust Extinction Law, $R_V$, and the Angular Power Spectrum of Dust Clouds 
%in the Direction of Differentially-Reddened Globular Clusters.'' 
%\ 2004, AJ (in prep.)\\

%\noindent
%Melbourne,
%J.~\& Guhathakurta, P. ``Measuring the Angular Power Spectrum of Dust 
%Clouds in the Direction of Differentially-Reddened Globular Clusters.''
%\ 2003, Astrophysics of Dust, Estes Park, Colorado, 
%May 26 - 30, 2003.~Edited by Adolf N.~Witt.,  \\

%\noindent
%Melbourne, J.~\& 
%Salzer, J.~J. ''Metal Abundances of KISS Galaxies. I. Coarse Metal 
%Abundances and the Metallicity-Luminosity Relation.''\ 2002, AJ, 123, 2302 \\

%\noindent
%Melbourne, J., Sarajedini, A., Layden, 
%A., \& Martins, D.~H. ``CCD Photometry of the Globular Cluster NGC 4833 
%and Extinction Near the Galactic Plane.'' \ 2000, AJ, 120, 3127 \\

%\noindent
%Deustua, S.~et al.''Supernovae''\ 
%1995, IAU CIRC, 6270, 1 \\

%\noindent
%Wegner, G., Salzer, 
%J.~J., Jangren, A., Gronwall, C., \& Melbourne, J. ``Spectroscopy of KISS 
%Emission-Line Galaxy Candidates. I. MDM Observations.''\ 2003, AJ, 125, 2373 \\


\end{document}
% LocalWords:  UC Astro Salzer al AJ Guhathakurta Metallicity Sarajedini Layden
% LocalWords:  CCD NGC Deustua IAU CIRC Wegner Jangren Gronwall MDM
