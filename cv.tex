\documentclass[10pt]{cv}
 
\font\cap=cmcsc10

\topmargin 0pt
\headheight 0pt
\headsep 0pt
\pagestyle{empty}
\parindent 0pt
\parskip \baselineskip
\oddsidemargin -0.35in
\evensidemargin -0.35in

\textwidth 7.1in

\setlength{\marginparsep}{0pt}
\setlength{\topmargin}{-0.1in}
\setlength{\headheight}{0in}
\setlength{\headsep}{0in}
\setlength{\topskip}{0.4in}
\setlength{\textheight}{9.3in}

\pagestyle{myheadings}

\usepackage{fancyhdr}
\pagestyle{fancy}
\setlength\headsep{0.2in}
\setlength\topmargin{-0.4in}
\setlength\headheight{0.2in}
\cfoot{\thepage/8}
\fancyhead[l]{Jason Melbourne}
\fancyhead[r]{Curriculum Vitae}
\input{/Users/jmel/bib/def.tex}
%set the number in the second set of curly braces to the number you want displayed on
%the first page
\setcounter{page}{1}
  
\begin{document}

\begin{center}
{\huge \textbf{\sc Jason Melbourne}}\\
\smallskip
{\Large{ Curriculum Vitae}}\\
%{\small {\em Updated \today}}\\
\rule{18cm}{0.5mm}\\
\end{center}
\normalsize

\addresses
{
{\bf Office Address}\\
California Institute of Technology\\
MS 301-17\\
Pasadena, CA 91125 \\
}
{
{\bf Contact Information}\\
Phone: (831) 332-4607\\
E-mail: jmel@caltech.edu\\
Web: http://www.submm.caltech.edu/$\sim$jmel/
}

\begin{llist}

\vspace{0.1in}

\sectiontitle{Education}
\employer{Doctor of Philosophy, Astronomy \& Astrophysics}
\location{May 2006}
University of California, Santa Cruz\\
Thesis:  ``The Optical and Infrared Evolution of Galaxies'' (Advisors: C. Max \& D. C. Koo) 
\employer{Master of Arts, Astronomy}
\location{May 2001}
Wesleyan Univeristy\\
Thesis: ``Metal Abundances in KISS Galaxies'' (Advisor J. J. Salzer)

\employer{Bachelor of Arts, Physics \& Astronomy}
\location{May 1995}
University of California, Berkeley

\vspace{0.1in}

\thispagestyle{empty}
\sectiontitle{Appoint-\\ments}

{\sc Postdoctoral Fellow}, Physics Department, California Institute of Technology  
\location{2007-- }
\vspace{-0.1in}\\
{\sc Postdoctoral Fellow}, Center for Adaptive Optics, UC Santa Cruz
\location{2006--2007}

\vspace{-0.1in}
\sectiontitle{Grants}
{\sc Transforming Undergraduate STEM Experiences \\ Through the Next Generation of Scientist and Engineer Educators}
\location{2012 - 2015}
NSF, Department of Undergraduate Education, Collaborator \$600,000
\vspace{0.05in}\\
{\sc Panchromatic Hubble Andromeda Treasury}
\location{2010 - 2012}
HST Multi-cycle Treasury Program, Caltech P.I. of sub-award \$93,000
\vspace{0.05in}\\
{\sc The Sites and Triggers of Star Formation in Large Disk Galaxies Since $z=1$}
\location{2008}
HST Archival Grant, AR-10965. P.I. \$53,000

\vspace{0.1in}

\sectiontitle{Major\\ Research\\ Programs}

{\sc Asymptotic Giant Branch (AGB) Stars}
\location{2009--present}
A study of AGB stars in nearby galaxies and star clusters. I showed the AGB contributes a large fraction of the near-infrared light of galaxies, but that recent models over-predict their contribution. 
 \vspace{-0.1in}\\
{\sc Dust Obscured Galaxies}
\location{2007--present}
A study of ultra-luminous infrared galaxies in the early universe.  I led multiple papers on morphologies, black hole masses, and far-infrared dust properties of these extreme galaxies. I also organize our multi-institutional team meetings, write observing proposals, and plan future science.     
\vspace{0.05in}\\
{\sc Luminous Infrared Galaxies (LIRGs)}
\location{2005--2007}
Morphologies and stellar populations of $z=1$ LIRGs. I showed that the high star formation rates in $z=1$ galaxies were occurring, primarily, in morphologically normal disks.  My results suggest that the decline in star formation since $z=1$ has been driven by gas depletion rather than a drop in merger rates, an alternative hypothesis.
 \vspace{0.05in}\\
{\sc Center for Adaptive Optics Treasury Survey}
\location{2005--2007}
High spatial resolution near-infrared imaging of 300 $z\sim1$ galaxies from Keck adaptive optics (AO). I developed AO observing techniques, built the first online AO database, and led papers on high-z supernova, galaxy mergers, and infrared luminous galaxies. 
 \vspace{0.05in}\\
 \clearpage
{\sc Metallicities of Dwarf Galaxies}
\location{2000--2005}
Oxygen abundance of hundreds of low redshift dwarf galaxies.  First to show that the metallicity-luminosity relation holds over roughly 10 magnitudes in luminosity. 

\vspace{0.08in}


\sectiontitle{Instrumen-\\tation}

{\sc NIRES, A Near-Infrared Spectrograph for Keck Observatory}
\location{2010--present}
I led the detector characterization, and showed that it would not achieve the required performance. With the University of Toronto, we have purchased a new detector. (P.I. Keith Matthews)
\vspace{0.05in}\\
{\sc Thirty Meter Telescope (TMT) Site Testing}
\location{2004}
I participated in a TMT site testing campaign on Mauna Kea.  I assisted in deployment of kite and balloon born instrumentation, and helped synthesize the results.  (P.I. Mark Chun)

\sectiontitle{Selected\\ Successful\\Proposals \& \\ Observing}
%\markright{{Jason Melbourne, Curriculum Vitae}
%\hspace{10mm}}

{\sc The Starburst--AGN Connection, P.I.}
\location{2012}
Keck Laser Guide Star Adaptive Optics Integral Field Spectroscopy of Nearby AGN
\vspace{0.05in}\\
{\sc   The Local Group Infrared Cluster Survey, P.I.}
\location{ 2011}
Keck Laser Guide Star Adaptive Optics Imaging with NIRC2
\vspace{0.05in}\\
{\sc   The Panchromatic Hubble Andromeda Treasry, Co.I.}
\location{2010}
HST Multi-cycle Treasury Program for Multi-band Imaging Across Andromeda Galaxy.
\vspace{0.05in}\\
{\sc  Rest-frame Optical Spectroscopy of $z=2$ Dust Obscured Galaxies, P.I. }
\location{2009}
Palomar Near-IR Spectroscopy with TripleSpec\\ Keck Integral-Field Spectroscopy with OSIRIS
\vspace{0.05in}\\
{\sc  Resolved Stellar Populations of Nearby Galaxies, P.I.}
\location{2008}
Keck Laser Guide Star Adaptive Optics Imaging with NIRC2
\vspace{0.05in}\\
{\sc  A Calibration Database for Stellar Models of AGB Stars, Co.I.}
\location{2008}
HST Near-IR Imaging with WFC3, SNAP-11719
\vspace{0.05in}\\
{\sc The Distribution of Warm Dust in Luminous Infrared Galaxies, P.I.}
\location{2008--2009}
Gemini Mid-IR Imaging with TReCS
\vspace{0.05in}\\
{\sc  The Morphologies of $z=2$ Dust Obscured Galaxies, Co.I.}
\location{2007--2008}
Keck Laser Guide Star Adaptive Optics Imaging with NIRC2
\vspace{0.05in}\\
{\sc  Rest-frame Optical Morphologies of Galaxies in GOODS, Co.I.}
\location{2004--2006}
Keck Laser Guide Star Adaptive Optics Imaging with NIRC2


\sectiontitle{Teaching\\Experience}

\employer{Adjunct Professor, Pomona College }
\location{2010--2011}
Introductory Physics Laboratory.  I collaborated on the design of 20 labs in Mechanics and E\&M. My focus was on defining content and process goals. I designed and taught a mini-lecture at the start of each lab
\vspace{-0.05in}
\employer{Visiting Professor, Caltech}
\location{2010}
Taught two lectures of graduate level Stellar Structure and Evolution.
\vspace{-0.05in}
\employer{Director, Think Like an Astronomer}
\location{2010}
In an effort to promote science literacy, I have developed a 5-session astronomy short course for the general public.  This course introduces major themes in astronomy through inquiry-based activities, and exquisite astronomical images.  I taught this course at Glendale Community College.
\vspace{-0.05in}
\employer{Adjunct Professor, Mt San Antonio Community College}
\location{2009}
Introductory Astronomy Lecture and Lab. I developed and taught the curriculum, for this lecture and lab course, including a semester long Sun tracking lab, and a galaxy morphology inquiry activity.
\vspace{-0.15in}
\employer{Lead Instructor, The Center for Adaptive Optics Short Course}
\location{2005}
A one-week course on astronomy, light, and optics to prepare undergraduates for summer research positions in the Center for Adaptive Optics.
\vspace{-0.05in}
\employer{Teaching Assistant, University of California Santa Cruz }
\location{2001--2002}

\vspace{-0.1in}

\sectiontitle{Selected\\Students\\Advised}
{\sc Gautam Upadhya, Caltech Summer Undergraduate Research Fellow } 
\location{2011--present}
Tracking the Evolved Stars of M33 with Adaptive Optics Imaging in the Near-IR
\vspace{0.05in}\\
{\sc Hanae Inami, Graduate Student at the Spitzer Science Center } 
\location{2010--present}
Integral field spectroscopy of low redshift Luminous Infrared Galaxies
\vspace{0.05in}\\
{\sc John Forbes, Caltech Undergraduate} 
\location{2009--2010}
Spatially Resolved Stellar Populations of $z=1$ Luminous Infrared Galaxies
\vspace{0.05in}\\
{\sc Abhiram Chivikula, Caltech Summer Undergraduate Research Fellow } 
\location{2008}
Spectral Energy Distributions of High Redshift Dust-Obscured Galaxies

\vspace{0.08in}


\sectiontitle{Selected\\ Service}

{\sc Referee}
\location{2005--present}
The Astronomical Journal, The Astrophysical Journal, and \\Publications of the Astronomical Society of the Pacific
\vspace{0.05in}\\
{\sc Science Organizing Committee}
\location{2010}
Massive Galaxies III, Tucson, Arizona 
\vspace{0.05in}\\
{\sc NSF Panel Member} 
\location{2007-2008} 
Astronomy and Astrophysics Research Grants
\vspace{0.05in}\\
{\sc Director and Lead Instructor}
\location{2006, 2008} 
The Keck Adaptive Optics Workshop, University of California Santa Cruz
\vspace{0.05in}\\
{\sc Science Advisor, The California Math and Science Project}
\location{2006} 
Watsonville California School District\\

\vspace{-0.2in}


\sectiontitle{Selected\\Recent \\Talks}

{\sc Conference: Through the Infrared Looking Glass, Pasadena CA }
\location{September 2011} 
``The Far-IR Spectral Energy Distributions of $z=2$ Dust Obscured Galaxies''
\vspace{0.05in}\\
{\sc Conference: The Starbust-AGN Connection: Madrid, Spain}
\location{September 2011} 
``Black Hole Masses and Star Formation Rates of $z=2$ Dust Obscured Galaxies\\ Revealed with Keck OSIRIS Integral Field Spectroscopy''
\vspace{0.05in}\\
{\sc Colloquium: UC San Diego, UCLA, UC Irvine, Columbia, Harvard}
\location{March--May 2011} 
``Infrared Luminosities of AGB and RHeB Stars from HST WFC3: Implications \\for Measuring Stellar Masses of Galaxies''
\vspace{0.05in}\\
{\sc Conference: Why Do Galaxies Care About AGB Stars? Vienna }
\location{September 2010} 
``The Contribution of Asymptotic Giant Branch Stars to the Infrared Luminosities \\of Galaxies: Implications for Measuring Stellar Masses of Galaxies''
\vspace{0.05in}\\
{\sc Colloquium: University of Hawaii }
\location{September 2010} 
``Dust Obscured Galaxies at $z=2$''
\vspace{0.05in}\\
{\sc Colloquium: UC Santa Cruz }
\location{April 2010} 
``The Local Group Infrared Cluster Survey''
\vspace{0.05in}\\
{\sc Conference: Dwarf Galaxies in the Local Universe, SAO, Russia}
\location{September 2009} 
``Asymptotic Giant Branch Stars as Probes of Star Formation History''

\sectiontitle{Selected\\Collabora-\\tions}

{\sc Panchromatic Hubble Andromeda Treasury (PHAT)}
\location{P.I. Dalcanton}
\vspace{-0.1in}\\
{\sc Local Group Infrared Cluster Survey (LoGICS)}
\location{P.I. Melbourne}
\vspace{-0.1in}\\
{\sc NOAO Deep Wide Field Survey (NDWFS)}
\location{P.I.s Dey, Januzzi}
\vspace{-0.1in}\\
{\sc Great Observatories All-sky LIRG Survey (GOALS)}
\location{P.I.s Sanders, Armus}
\vspace{-0.1in}\\
{\sc Center for Adaptive Optics Treasury Survey (CATS)}
\location{P.I.s Larkin, Max, Koo}

\sectiontitle{References}

{\sc Dr. B. T. Soifer --- Director, Spitzer Science Center, \\
Department Chair, Physics, Math and Astronomy}\\
California Institute of Technology\\ 
MC 301-17\\
1200 E. California Blvd.\\
Pasadena, CA 91125\\
(626)-395-4241  bts@ipac.caltech.edu

{\sc Dr. Claire Max --- Director, Center for Adaptive Optics}\\
UCO/Lick Observatories\\
University of California Santa Cruz\\
1156 High St.\\
Santa Cruz, CA 95064\\
(831)-459-2049  max@ucolick.org

{\sc Dr. Julianne Dalcanton --- Professor}\\
Department of Astronomy \\
University of Washington, Box 351580 \\
Seattle WA 98195-1580\\
(206) 685-2155 jd@astro.washington.edu

%{\sc Dr. Thomas Moore � Professor}\\
%Department of Physics\\
%Pomona College\\
%610 N College Ave\\
%Claremont, CA 91711\\
%(909)-621-8726  TAM04747@pomona.edu

{\sc Dr. James Larkin --- Professor}\\
Department of Physics and Astronomy\\
University of California, Los Angeles\\
430 Portola Plaza\\
Box 951547\\
Los Angeles, CA. 90095-1547\\
(310) 825-9470  larkin@astro.ucla.edu

{\sc Dr. Lee Armus --- Staff Scientist}\\
Infrared Processing and Analysis Center\\
California Institute of Technology\\
MC 220-6 \\
Pasadena, CA 91125\\
(626) 395-8569  lee@ipac.caltech.edu

\sectiontitle{Publication\\Summary}

{\sc First-Author Publications in Review}
\location{1}
\vspace{-0.1in}\\
{\sc First-Author Refereed Publications}
\location{16}
\vspace{-0.1in}\\
{\sc Additional Refereed Publications}
\location{31}
\vspace{-0.1in}\\
{\sc First-Author Citation Count, November 2012}
\location{346}
\vspace{-0.1in}\\
{\sc Total Citation Count,  November 2012}
\location{1252}

\vspace{-0.1in}

\sectiontitle{First-\\Author\\Publications\\in Review}
{\sc The Contribution of Thermally-Pulsing Asymptotic Giant Branch and Red Super Giant Stars to the Luminosities of the Magellanic Clouds at $1-24$~\um}\\
{\bf Melbourne, J.} \& Boyer, M. 2012. Astrophysical Journal, in review


\sectiontitle{First-\\Author\\Refereed\\Publications}

{\sc The Spectral Energy Distributions and Infrared Luminosities of Dust Obscured Galaxies from Herschel and Spitzer, Not Your Average $z=2$ ULIRGs}\\
{\bf Melbourne, J.}, Soifer, B.T., Desai, V., Pope A., Armus L., Dey, A., Bussmann R.S.,  Jannuzi, B.T., Alberts, S. 2012. The Astronomical Journal, Volume 143, Issue 5, article id. 125 

{\sc The Contribution of TP-AGB and RHeB Stars to the Near-IR Luminosity of Local Galaxies: Implications for Stellar Mass Measurements of High Redshift Galaxies} \\
{\bf Melbourne, J.}, Williams, B., Dalcanton, J., Rosenfield, P.,  Girardi, L., Marigo, P., Weisz, D., Dolphin A., Boyer, M., Olsen, K., Skillman E., Seth, A. 2012. The Astrophysical Journal, Volume 748, Issue 1, article id. 47

{\sc The Black Hole Masses and Star Formation Rates of $z >1$ Dust Obscured Galaxies (DOGs): Results from Keck OSIRIS Integral Field Spectroscopy}\\
{\bf Melbourne, J.}, Peng, Chien Y, Soifer, B.T., Desai, V., Armus, L., Bussmann, R.S., Dey A., Matthews, K. 2011. The Astronomical Journal, Volume 141, Issue 4,           pp. 141--152

{\sc The Asymptotic Giant Branch and the Tip of the Red Giant Branch as Probes of Star Formation History: The Nearby Dwarf Irregular Galaxy KKH 98}\\
{\bf Melbourne, J.}, Williams, B., Dalcanton, J., Ammons, S.M., Max, C.,  Koo, D.C., Girardi, L., Dolphin, A. 2010. The Astrophysical Journal, Volume 712, Issue 1, pp. 469--483

{\sc High Redshift Dust Obscured Galaxies, a Morphology-SED Connection Revealed by Keck Adaptive Optics}\\
{\bf Melbourne, J.}, Bussman, S., Brand, K., Desai, V., Armus, L., Dey, A., Jannuzi,  B. T., Houck, J. R.,  Soifer, B. T., Matthews, K. 2008. The Astronomical Journal, Volume 137, Issue 6, pp. 4854--4866  

{\sc Morphologies of High-Redshift Dust-Obscured Galaxies from Keck Laser Guide Star Adaptive Optics}\\
{\bf Melbourne, J.}, Desai, V., Armus, Lee, Dey, Arjun, Brand, K., Thompson, D., Soifer, B. T., Matthews, K., Jannuzi, B. T., Houck, J. R., 2008. The Astronomical Journal, Volume 136, Issue 3, pp. 1110-1117

{\sc Triggered or Self-Regulated Star Formation Within Intermediate Redshift Luminous Infrared Galaxies. I. Morphologies and Spectral Energy Distributions}\\
{\bf Melbourne, J.}, Ammons, M., Wright, S. A., Metevier, A., Steinbring, E., Max, C., Koo, D. C., Larkin, J. E., Barczys, M., 2008. The Astronomical Journal, Volume 135, Issue 4, pp. 1207--1224

{\sc Rest-Frame R-band Lightcurve of a $z \sim1.3$ Supernova Obtained with Keck Laser Adaptive Optics}\\
{\bf Melbourne, J.}, Dawson K., Koo D. C.., Max C., Larkin J. Wright S.,Steinbring E., Barczys, M., Aldering, G., Barbary, K., Doi, M., Fadeyev, V., Goldhaber, G., Hattori, T., Ihara Y., Kashikawa N., Konishi K., Kowalski M., Kuznetsova, N., Lidman, C., Morokuma, T., Perlmutter, S., Rubin D., Schlegel D., Spadafora, A. L., Takanashi, N., Yasuda, N. 2007. The Astronomical Journal. Volume. 133, Issue 6, pp. 2709--2715

{\sc Radius Dependent Luminosity Evolution of Blue Galaxies in GOODS-N}\\
{\bf Melbourne, J.}, Phillips, A., Harker, J., Novak, G., Koo, D. C., Faber, S. M., 2007. The Astrophysical Journal. Volume 660, Issue 1, pp. 81--96

{\sc Exploring the Optical and Infrared Evolution of Galaxies}\\
{\bf Melbourne, J.}, 2006. Ph.D. Thesis, UC Santa Cruz

{\sc Optical Morphology Evolution of Infrared Luminous Galaxies in GOODS-N}\\
{\bf Melbourne, J.}, Koo, D. C., Le Floc�h, E., 2005. The Astrophysical Journal. Volume 632, Issue 2, pp. L65--L68

{\sc Merging Galaxies in GOODS-S: First Extragalactic Results from Keck Laser Adaptive Optics}\\
{\bf Melbourne, J.}, Wright, S. A., Barczys, M., Bouchez, A. H., Chin, J., van Dam, M. A., Hartman, S., Johansson, E., Koo, D. C., Lafon, R., Larkin, J., Le Mignant, D., Lotz, J., Max, C. E., Pennington, D. M., Stomski, P. J., Summers, D., Wizinowich, P. L., 2005. The Astrophysical Journal. Volume 625, Issue 1, pp. L27--L30

{\sc Measuring the Slope of the Dust Extinction Law and the Power Spectrum of Dust Clouds Using Differentially Reddened Globular Clusters}\\
{\bf Melbourne, J.}, Guhathakurta, P., 2004. The Astronomical Journal. Volume 128, Issue 1, pp. 271--286

{\sc Metal Abundances of KISS Galaxies. II. Nebular Abundances of 12 Low-Luminosity Emission-Line Galaxies}\\
{\bf Melbourne, J.}, Phillips, A., Salzer, John J., Gronwall, C., Sarajedini, Vicki L., 2004. The Astronomical Journal. Volume 127, Issue 2, pp. 686--703

{\sc Metal Abundances of KISS Galaxies. I. Coarse Metal Abundances and the Metallicity-Luminosity Relation}\\
{\bf Melbourne, J.}, Salzer, John J., 2002. The Astronomical Journal. Volume 123,      Issue 5, pp. 2302--2311

{\sc CCD Photometry of the Globular Cluster NGC 4833 and Extinction Near the Galactic Plane}\\
{\bf Melbourne, J.}, Sarajedini, Ata, Layden, Andrew, Martins, Donald H., 2000. The Astronomical Journal. Volume 120, Issue 6, pp. 3127--3138

\sectiontitle{Other\\Refereed\\Publications}

{\sc AGN Unification at $z\sim1$: $u - R$ Colors and Gradients in X-Ray AGN Hosts}\\
Ammons, S. Mark, Rosario, David J. V., Koo, David C., Dutton, Aaron A., {\bf Melbourne, J.}, Max, Claire E., Mozena, Mark, Kocevski, Dale D., McGrath, Elizabeth J., Bouwens, Rychard J., Magee, Daniel K. 2011. The Astrophysical Journal, Volume 740, Issue 1, pp. 3

{\sc The Nuclear Structure in Nearby Luminous Infrared Galaxies: Hubble Space Telescope NICMOS Imaging of the GOALS Sample}\\
Haan, S., Surace, J. A., Armus, L., Evans, A. S., Howell, J. H., Mazzarella, J. M., Kim, D. C., Vavilkin, T., Inami, H., Sanders, D. B., Petric, A., Bridge, C. R., {\bf Melbourne, J.}, Charmandaris, V., Diaz-Santos, T., Murphy, E. J., U, V., Stierwalt, S., Marshall, J. A. 2011. The Astronomical Journal, Volume 141, Issue 3, pp. 100 

{\sc Hubble Space Telescope Morphologies of $z\sim2$ Dust-obscured Galaxies. II. Bump Sources}\\
Bussmann, R. S., Dey, A., Lotz, J., Armus, L., Brown, M., Desai, V., Eisenhardt, P., Higdon, J., Higdon, S., Jannuzi, B. T., Le Floc'h, E., {\bf Melbourne, J.}, Soifer, B. T., Weedman, D. 2011. The Astrophysical Journal, Volume 733, Issue 1, pp. 21

{\sc Active Disk Building in a Local H I-massive LIRG: The Synergy Between Gas, Dust, and Star Formation}\\
Cluver, M. E., Jarrett, T. H., Kraan-Korteweg, R. C., Koribalski, B. S., Appleton, P. N., {\bf Melbourne, J.}, Emonts, B., Woudt, P. A. 2010. The Astrophysical Journal, Volume 725, Issue 2, pp. 1550-1562

{\sc The ACS Nearby Galaxy Survey Treasury. IX. Constraining Asymptotic Giant Branch Evolution with Old Metal-poor Galaxies}\\
Girardi, L�o, Williams, Benjamin F., Gilbert, Karoline M., Rosenfield, Philip, Dalcanton, Julianne J., Marigo, Paola, Boyer, Martha L., Dolphin, Andrew, Weisz, Daniel R., {\bf Melbourne, J.}, Olsen, Knut A. G., Seth, Anil C., Skillman, Evan. 2010. The Astrophysical Journal, Volume 724, Issue 2, pp. 1030-1043

{\sc The Buried Starburst in the Interacting Galaxy II Zw 096 as Revealed by the Spitzer Space Telescope}\\
Inami, H., Armus, L., Surace, J. A., Mazzarella, J. M., Evans, A. S., Sanders, D. B., Howell, J. H., Petric, A., Vavilkin, T., Iwasawa, K., Haan, S., Murphy, E. J., Stierwalt, S., Appleton, P. N., Barnes, J. E., Bothun, G., Bridge, C. R., Chan, B., Charmandaris, V., Frayer, D. T., Kewley, L. J., Kim, D. C., Lord, S., Madore, B. F., Marshall, J. A., Matsuhara, H., {\bf Melbourne, J.}, Rich, J., Schulz, B., Spoon, H. W. W., Sturm, E., U, V., Veilleux, S., Xu, K. 2010. The Astronomical Journal, Volume 140, Issue 1, pp. 63-74

{\sc The Great Observatories All-sky LIRG Survey: Comparison of Ultraviolet and Far-infrared Properties}\\
Howell, Justin H., Armus, Lee, Mazzarella, Joseph M., Evans, Aaron S., Surace, Jason A., Sanders, David B., Petric, Andreea, Appleton, Phil, Bothun, Greg, Bridge, Carrie, Chan, Ben H. P., Charmandaris, Vassilis, Frayer, David T., Haan, Sebastian, Inami, Hanae, Kim, Dong-Chan, Lord, Steven, Madore, Barry F., {\bf Melbourne, J.}, Schulz, Bernhard, U, Vivian, Vavilkin, Tatjana, Veilleux, Sylvain, Xu, Kevin. 2010. The Astrophysical Journal, Volume 715, Issue 1, pp. 572-588

{\sc Infrared Luminosities and Dust Properties of $z\sim2$ Dust-obscured Galaxies}\\
Bussmann, R. S., Dey, Arjun, Borys, C., Desai, V., Jannuzi, B. T., Le Floc'h, E., {\bf Melbourne, J.}, Sheth, K., Soifer, B. T. 2009. The Astrophysical Journal, Volume 705, Issue 1, pp. 184-198

{\sc Strong Polycyclic Aromatic Hydrocarbon Emission from $z\sim 2$ ULIRGs}\\
Desai, Vandana, Soifer, B. T., Dey, Arjun, LeFloc'h, Emeric, Armus, Lee, Brand, Kate, Brown, Michael J. I., Brodwin, Mark, Jannuzi, Buell T., Houck, James R., Weedman, Daniel W., Ashby, Matthew L. N., Gonzalez, Anthony, Huang, Jiasheng, Smith, Howard A., Teplitz, Harry, Willner, Steve P., {\bf Melbourne, J.} 2009. The Astrophysical Journal, Volume 700, Issue 2, pp. 1190-1204

{\sc GOALS: The Great Observatories All-Sky LIRG Survey}\\
Armus, L., Mazzarella, J. M., Evans, A. S., Surace, J. A., Sanders, D. B., Iwasawa, K., Frayer, D. T., Howell, J. H., Chan, B., Petric, A., Vavilkin, T., Kim, D. C., Haan, S., Inami, H., Murphy, E. J., Appleton, P. N., Barnes, J. E., Bothun, G., Bridge, C. R., Charmandaris, V., Jensen, J. B., Kewley, L. J., Lord, S., Madore, B. F., Marshall, J. A., {\bf Melbourne, J.}, Rich, J., Satyapal, S., Schulz, B., Spoon, H. W. W., Sturm, E., U, V., Veilleux, S., Xu, K. 2009. Publications of the Astronomical Society of the Pacific, Volume 121, issue 880, pp.559-576

{\sc Hubble Space Telescope Morphologies of $z \sim 2$ Dust Obscured Galaxies. I. Power-Law Sources}\\
Bussmann, R. S., Dey, Arjun, Lotz, J., Armus, L., Brand, K., Brown, M. J. I., Desai, V., Eisenhardt, P., Higdon, J., Higdon, S., Jannuzi, B. T., LeFloc'h, E., {\bf Melbourne, J.}, Soifer, B. T., Weedman, D. 2009. The Astrophysical Journal, Volume 693, Issue 1, pp. 750-770

{\sc The Origin of the 24 $\mu$m Excess in Red Galaxies}\\
Brand, Kate, Moustakas, John, Armus, Lee, Assef, Roberto J., Brown, Michael J. I., Cool, Richard R., Desai, Vandana, Dey, Arjun, LeFloc'h, Emeric, Jannuzi, Buell T., Kochanek, Christopher S., {\bf Melbourne, J.}, Papovich, Casey J., Soifer, B. T. 2009. The Astrophysical Journal, Volume 693, Issue 1, pp. 340-346

{\sc Spatially Resolved Stellar Populations of Eight GOODS-South AGN at $z\sim1$}\\
Ammons, M., {\bf Melbourne, J.}, Max, C., Koo, D., Rosario, D. 2009. The Astronomical Journal.  Volume 137, Issue 1, pp. 470-497

{\sc Cats: Optical to Near-Infrared Colors of the Bulge and Disk of Two $z = 0.7$ Galaxies Using Hubble Space Telescope and Keck Laser Adaptive Optics Imaging}\\
Steinbring, E., {\bf Melbourne, J.}, Metevier, A. J., Koo, D. C., Chun, M. R., Simard, L., Larkin, J. E., Max, C. E., 2008. The Astronomical Journal, Volume 136, Issue 4, pp. 1523-1532

{\sc Superresolving Distant Galaxies with Gravitational Telescopes: Keck Laser Guide Star Adaptive Optics and Hubble Space Telescope Imaging of the Lens System SDSS J0737+3216}\\
Marshall, P., Treu, T., {\bf Melbourne, J.}, Gavazzi, R., Bundy, K., Ammons, S.M. , Bolton, A., Burles, S., Larkin, J., Le Mignant, D., Koo, D., Koopmans, L�on V. E., Max, C., Moustakas, L.A., Steinbring, E., Wright, S.A., 2007. The Astrophysical Journal, Volume 671, Issue 2, pp. 1196-1211

{\sc A Survey of Galaxy Kinematics to $z\sim1$ in the TKRS/GOODS-N Field. II. Evolution in the Tully-Fisher Relation}\\
Weiner, Benjamin J., Willmer, Christopher N. A., Faber, S. M., Harker, Justin, Kassin, Susan A., Phillips, Andrew C., {\bf Melbourne, J.}, Metevier, A. J., Vogt, N. P., Koo, D. C., 2006. The Astrophysical Journal, Volume 653, Issue 2, pp. 1049-1069

{\sc A Survey of Galaxy Kinematics to $z\sim1$ in the TKRS/GOODS-N Field. I. Rotation and Dispersion Properties}\\
Weiner, B., Willmer, C., Faber, S. M., {\bf Melbourne, J.}, Kassin, S., Phillips, A., Harker, J., Metevier, A. J., Vogt, N. P., Koo, D. C., 2006. The Astrophysical Journal, Volume 653, Issue 2, pp. 1027-1048

{\sc Luminous Compact Blue Galaxies up to $z\sim1$ in the Hubble Space Telescope Ultra Deep Field. I. Small Galaxies or Blue Centers of Massive Disks?}\\
Noeske, K. G., Koo, D. C., Phillips, A. C., Willmer, C. N. A., {\bf Melbourne, J.}, Gil de Paz, A., Papaderos, P., 2006. The Astrophysical Journal. Volume 640, Issue 2,    pp. L143-L146

{\sc Metal Abundances of KISS Galaxies. IV. Galaxian Luminosity-Metallicity Relations in the Optical and Near-Infrared}\\
Salzer, J., Lee, J., {\bf Melbourne, J.}, Hinz, J., Alonso-Herrero, A., Jangren, A. 2005. The Astrophysical Journal. Volume 624,    Issue 2, pp. 661-679

{\sc Metal Abundances of KISS Galaxies. III. Nebular Abundances for Fourteen Galaxies and the Luminosity-Metallicity Relationship for H II Galaxies}\\
Lee, J., Salzer, J., {\bf Melbourne, J.} 2004. The Astrophysical Journal. Volume 616, Issue 2, pp. 752-767

{\sc The Host Galaxy of GRB 031203: Implications of Its Low Metallicity, Low Redshift, and Starburst Nature}\\
Prochaska, J., Bloom, J., Chen, H., Hurley, Kevin C., {\bf Melbourne, J.}, Dressler, A.,  Graham, J., Osip, D., Vacca, W. D., 2004. The Astrophysical Journal. Volume 611, Issue 1, pp. 200-207

{\sc Spectroscopy of KISS Emission-Line Galaxy Candidates. I. MDM Observations}\\
Wegner, Gary, Salzer, John J., Jangren, Anna, Gronwall, Caryl, {\bf Melbourne, J.} 2003. The Astronomical Journal. Volume 125, Issue 5, pp. 2373-2392



\end{llist}

\vspace{1.2cm}
\footnotesize
%\noindent \hspace{140mm}{\em Last updated \today .}\normalsize

\end{document}
 



